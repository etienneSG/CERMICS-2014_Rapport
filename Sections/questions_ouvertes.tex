\chapter{Questions ouvertes}

Le problème initial était de savoir si la résolution du SSBP dans le cas d'un graphe circulaire pouvait s'effectuer en temps polynomial. Nous n'avons pas pu répondre à cette question dans le cas général. En revanche, nous avons obtenu des résultats dans quelques cas particuliers et laissé quelques questions ouvertes.

\section{Capacité infinie}

Dans le cas de la capacité infinie (c'est-à-dire une capacité de transport supérieure au nombre total de vélos sur le graphe), nous vons montré que si les points de départ et d'arrivée étaient les mêmes, alors le problème se résolvait en temps polynomial. En revanche, lorsque les points de départ et d'arrivée sont différents, la question reste ouverte.

\section{Capacité unitaire}

Pour la capacité unitaire, nous avons fait des hypothèses très fortes. Nous avons supposé que les points de départ et d'arrivée étaient identiques et que les stations étaient de la forme $(1,0)$ ou $(0,1)$. Dans ce cas, nous avons montré que si la conjecture \ref{conj: capacité unitaire - un passage} est vraie, alors le problème se résout en temps polynomial. La véracité de cette conjecture reste à démontrer.

Il reste également à savoir si, en enlevant l'hypothèse sur la forme des stations ou en enlevant l'hypothèse sur le point départ et d'arrivée, la résolution peut s'effectuer en temps polynomial.

\section{Cas général du graphe circulaire}

Le dernier point concerne le cas général du graphe circlaire. La complexité de la résolution du SSBP sur un graphe circulaire quelconque reste ouverte.