\chapter*{Organisme d'accueil : le CERMICS}
\addcontentsline{toc}{chapter}{Organisme d'accueil : le CERMICS}

Le CERMICS (Centre d'Enseignement et de Recherche en Mathématiques et Calcul Scientifique) est un laboratoire de mathématiques appliquées de l'\'Ecole des Ponts ParisTech et de l'Université Paris-Est. Ses activités sont regroupées en trois grands domaines : le calcul scientifique, l'optimisation et les probabilités appliquées.

L'équipe de cacul scientifique s'intéresse aux applications liées à la convection des polluants dans le sol et à l'hydrologie (souterraine ou de surface). Dans ce cadre, elle développe des méthodes numériques basées sur les éléments finis. En plus de ces applications, elle explore une grande variété de sujets liés à la modélisation et à la simulation numérique des phénomènes physiques telle la physique particulaire et multi-échelle. L'étude est à la fois théorique (sur les propriétés des models) et pratique (sur l'implémentation algorithmique).

L'équipe d'optimisation étudie le contrôle des systèmes dynamiques et en particulier les méthodes d'optimisation numérique et stochastique. Elle étudie également les transports ou la gestion de ressources. Enfin, une partie est consacrée à la recherche opérationnelle sous la direction de Frédéric Meunier mon maître de stage. Ici, il s'agit de modéliser des problèmes industriels et d'y apporter des réponses théoriques et pratiques. La recherche peut également se faire sur des questions théoriques encore ouverte dans la recherche opérationnelle.

Pour terminer, en probabilités appliquées sont étudiées les méthodes numériques et probabilistes de représentation des solutions des équations aux dérivées partielles, la modélisation probabiliste de manière générale ou en lien avec la physique, la biologie et les mathématiques financières.
