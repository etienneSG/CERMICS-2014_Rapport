\chapter{Capacité de transport unitaire}
\label{Capacite unitaire}

\section{Hypothèses}
\label{Capacité unitaire - hypothèse}

La capacité de transport unitaire pourrait correspondre au cas d'autolib'. Ici, nous supposons que :
\begin{itemize}
\item la capacité est unitaire (ie. $C=1$).
\item le point de départ et d'arrivée sont les mêmes (ie. $p=q$)
\item toute station a soit un vélo en excès, soit un vélo en défaut (ie. pour tout $v \in V$, $(x_v,y_v)~=~(1,0)$ ou $(x_v,y_v)~=~(0,1)$).
\end{itemize}
On note $2n$ le nombre de sommets du graphe. ($n$ sommets de la forme~$(1,0)$ et $n$ de la forme~$(0,1)$.) Afin de simplifier la rédaction, nous nommerons :
\begin{itemize}
\item station \plus une station de la forme $(1,0)$.
\item station \moins une station de la forme $(0,1)$.
\end{itemize}

\section{Conjecture et algorithme}
\label{capacité unitaire - conjecture et algorithme}

Au vu des recherches et des exemples traités, nous pensons que le résultat suivant est vrai.

\begin{conj} \label{conj: capacité unitaire - un passage}
Sous les hypothèses de la section \ref{Capacité unitaire - hypothèse}, il existe un trajet optimal et une arête $e_0$ tels que le camion passe au plus une fois par $e_0$ au cours de ce trajet.
\end{conj}

\begin{thm} \label{thm: capacité unitaire - optimalité}
Sous les hypothèses de la section~\ref{Capacité unitaire - hypothèse} et en supposant la conjecture~\ref{conj: capacité unitaire - un passage} vraie, l'algorithme suivant donne le trajet optimal du camion en temps polynomial.
\end{thm}
On note $S_{min}$ la meilleure solution réalisable obtenue et on l'initialise avec une solution faisant $2n$ fois le tour du graphe (donc de coût $2n\sum_{e \in E}c_e$).
\begin{enumerate}
\item \uline{Pour chaque $e_o \in E$,}
  \begin{enumerate}
  \item Construire le graphe linéaire $\bs{G(e_0)}$ obtenu en supprimant l'arrête $e_0$ de $G$.
  \item Calculer le coût $\Upsilon_{G(e_0)}$ de la solution optimale $S$ à l'aide de l'algorithme dans le cas linéaire.
  \item Si $\Upsilon_{G(e_0)} < \mbox{Coût}(S_{min})$, remplacer les caractéristiques de $S_{min}$ par celles de $S$ (c'est-à-dire son coût et l'arête $e_0$).
  \end{enumerate}
\item \uline{Pour chaque $e_o \in E$,}

  \uline{Pour chaque $u \in V$,}

  \uline{Pour chaque $v \in V$ tel que $v$ ne soit pas de la même forme que $u$},
  \begin{enumerate}
  \item Construire le graphe linéaire $\bs{G(e_0,u,v)}$ obtenu :
    \begin{itemize}
    \item en supprimant l'arrête $e_0$ de $G$.
    \item en prenant comme point de départ $v$.
    \item en prenant comme point d'arrivée $u$.
    \item Si $u$ est une station \plus et $v$ une station \moins, transformer $u$ en une station de la forme $(0,0)$ et $v$ en une station de la forme $(1,1)$.
    \end{itemize}
  \item Calculer le coût $\Upsilon_{G(e_0,u,v)}$ de la solution optimale à l'aide de l'algorithme dans le cas linéaire.
  \item Si $\Upsilon_{G(e_0,u,v)} + c(\tau_{u,v}) < \mbox{Coût}(S_{min})$, remplacer les caractéristiques de $S_{min}$ par celles de $S$ (c'est-à-dire son coût, l'arête $e_0$ et les sommets de départ $v$ et d'arrivée $u$).
  \end{enumerate}
\item \begin{enumerate}
\item Si la solution optimale est caractérisée par $e_0$, calculer chaque mouvement successif dans $\bs{G(e_0)}$ à l'aide de l'algorithme dans le cas linéaire.
  \item Si la solution optimale est caractérisée par $e_0$, $u$ et $v$, 
    \begin{enumerate}
    \item Calculer l'ensemble de la solution (donc des mouvements) à l'aide de l'algorithme dans le cas linéaire dans le graphe $\bs{G(e_0,u,v)}$.
    \item Ajouter le trajet $\tau_{u,v}$ de $u$ à $v$ en fin de solution.
    \item \begin{itemize}
      \item Si $p$ était initialement en excès, choisir un instant du cycle de la solution où le camion arrive vide en $p$ et repart plein et dérouler les mouvements de la solution.
      \item Si $p$ était initialement en défaut, choisir un instant du cycle de la solution où le camion arrive plein en $p$ et repart vide et dérouler les mouvements de la solution.
      \end{itemize}
    \end{enumerate}
  \end{enumerate}
\end{enumerate}


\section{Preuve de l'optimalité de l'algorithme}
\label{Capicité unitaire preuve}

La recherche de la solution optimale dans le graphe circulaire $G=(V,E)$ revient à chercher le plus court circuit hamiltonien dans le graphe biparti complet $G'=(V,E')$ défini et pondéré comme suit. Pour $u,v \in V$ avec $u \ne v$,
\begin{itemize}
\item si $u$ et $v$ sont toutes deux en excès ou toutes deux en défaut, on n'ajoute pas l'arête $(u,v)$.
\item sinon, on ajoute l'arête $(u,v)$ et on définit le coût $c_{(u,v)}$ de l'arête par $c_{(u,v)}~=~c(\tau_{u,v})$.
\end{itemize}
En effet, chaque vélo présent sur une station \plus ira équilibrer une station \moins. Pour une solution du SSBP, en marquant l'origine des vélos et en appliquant la règle d'une pile de vélos (ie. le dernier vélo déposé sur une station est le premier ramassé), on obtient un couplage entre les stations \plus et \moins et la solution dans le graphe $G'$. Réciproquement, une solution dans le graphe $G'$ donne aisément une solution dans le graphe $G$.

Comme la recherche du plus court circuit hamiltonien ne dépend pas du point de départ, on en déduit que la recherche d'une solution optimale au SSBP dans $G$ ne dépend pas du point de départ du camion.

\uline{$1^{\mbox{er}}$ cas} : \uline{il existe $e_0 \in E$ tel que le camion ne passe pas par $e_0$.}

Dans ce cas, la solution optimale est directement donnée par l'algorithme dans le cas linéaire appliqué au graphe $\bs{G(e_0)}$.
\\

\uline{$2^{\mbox{ème}}$ cas} : \uline{le camion passe au moins une fois par toutes les arêtes et il existe $e_0 \in E$ tel que le camion passe exactement une fois par $e_0$.}

\begin{itemize}
\item Si l'arête $e_0$ est traversée avec un vélo, ce vélo venait d'une station \plus $u$ et équilibrera une station \moins $v$.
\item Si l'arête $e_0$ est traversée sans vélo, le camion se contentait de passer d'une station \moins $u$ vers une station \plus $v$.
\end{itemize}
On suppose connus ces deux stations $u$ et $v$ et le mouvement du camion. Comme le camion ne repasse plus par $e_0$, il suffit d'équilibrer le graphe $\bs{G(e_0,u,v)}$. On sait que l'algorithme dans le cas linéaire donne une solution optimale $S(e_0,u,v)$. On en déduit donc une solution $S$ équilibrant le graphe $G$ en ajoutant à $S(e_0,u,v)$ le mouvement initial de $u$ à $v$.
\\

On a donc montré que l'on peut trouver en temps polynomial ($O(n^4)$) le coût de la solution optimale ainsi que ces éléments caractéristiques ($e_0$ si l'on ne passe jamais par $e_0$ ou $e_0$, $u$ et $v$ si l'on passe exactement une fois par $e_0$). Il reste à montrer que l'on peut calculer les mouvements en temps polynomial.

Dans le cas où la solution ne passe jamais par $e_0$, c'est le cas car on calcule la solution directement avec l'algorithme du cas linéaire.

Dans le cas où la solution passe exactement une fois par $e_0$ et où l'on connait $u$ et $v$, il est nécéssaire de calculer l'ensemble de la solution. On sait que la solution est réduisible à une alternance de stations \plus et \moins ($2n$ stations au total). En outre entre chaque station \plus et \moins ainsi trouvé, on parcourt au plus $2n-2$ stations. On a donc $O(n^2)$ premiers mouvements à calculer et à stocker. Chacun de ces mouvements se calculant en temps polynomial à l'aide de l'agorithme du cas linéaire, on peut donc calculer l'ensemble de la solution en $O(n^3)$ opérations.


\section{Cas où la solution optimale est bornée par $2\sum_{e \in E}c_e$}

\begin{prop}
Sous les hypothèses de la section~\ref{Capacité unitaire - hypothèse}, si le coût $\Upsilon_G$ de la solution optimale est strictement majoré par $2\sum_{e \in E}c_e$, alors l'algorithme de la section~\ref{capacité unitaire - conjecture et algorithme} donne la solution optimale et son coût en temps polynomial.
\end{prop}

\begin{proof}
Soit $S$ une solution optimale. Par l'absurde, si $S$ passe au moins deux fois par chaque arête, alors $\Upsilon_G~\ge~2\sum_{e \in E}c_e$, ce qui est contradictoire. Donc il existe une arête $e_0$ telle que le camion passe au plus une fois par $e_0$ au cours de ce trajet. Selon le théorème~\ref{thm: capacité unitaire - optimalité}, l'algorithme de la section~\ref{capacité unitaire - conjecture et algorithme} donne la solution optimale et son coût en temps polynomial.
\end{proof}

\begin{rmq}
La condition $\Upsilon_G~<~2\sum_{e \in E}c_e$ est une condition très forte. En effet, dans le pire des cas où les $n$ sommets $(1,0)$ suivent les $n$ sommets $(0,1)$, le coût $\Upsilon_G$ de la solution optimale est de l'ordre de $n^2\min_{e \in E}c_e$ alors que la condition précédente est linéaire en $n$.
\end{rmq}