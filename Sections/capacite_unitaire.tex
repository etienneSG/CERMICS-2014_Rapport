\chapter{Capacité du camion unitaire}
\label{Capacite unitaire}

\section{Hypothèses}
\label{Capacité unitaire - hypothèse}

La capacité du camion unitaire pourrait correspondre au cas d'autolib'. Ici, nous supposons que :
\begin{itemize}
\item la capacité est unitaire (ie. $C=1$).
\item le point de départ et d'arrivée sont les mêmes (ie. $p=q$)
\item toute station a soit un vélo en excès, soit un vélo en défaut (ie. pour tout $v \in V$, $(x_v,y_v)~=~(1,0)$ ou $(x_v,y_v)~=~(0,1)$).
\end{itemize}
On note $2n$ le nombre de sommet du graphe. ($n$ sommets de la forme~$(1,0)$ et $n$ de la forme~$(0,1)$.)

\section{Conjecture et algorithme}
\label{capacité unitaire - conjecture et algorithme}

Au vu des recherches et des exemples traité, nous pensons que le résultat suivant est vrai.

\begin{conj} \label{conj: capacité unitaire - un passage}
Sous les hypothèses de la section \ref{Capacité unitaire - hypothèse}, il existe un trajet optimal et une arête $e_0$ tels que le camion passe au plus une fois par $e_0$ au cours de ce trajet.
\end{conj}

\begin{thm} \label{thm: capacité unitaire - optimalité}
Sous les hypothèses de la section~\ref{Capacité unitaire - hypothèse} et en supposant la conjecture~\ref{conj: capacité unitaire - un passage} vraie, l'algorithme suivant donne le trajet optimal du camion en temps polynomial.
\end{thm}
\begin{enumerate}
\item \uline{Pour chaque $e_o \in E$,}

  \uline{Pour chaque $u \in V$,}

  \uline{Pour chaque $v \in V$ tel que $v$ ne soit pas de la même forme que $u$},
  \begin{enumerate}
  \item Construire le graphe linéaire $\bs{G(e_0,u,v)}$ obtenu :
    \begin{itemize}
    \item en supprimant l'arrête $e_0$ de $G$.
    \item en prenant comme point de départ $u$.
    \item en prenant comme point d'arrivée $v$.
    \item Transformer $u$ de la façon suivante
      \begin{itemize}
        \item Si $u$ est de la forme $(0,1)$, mettre $u$ de la forme $(0,0)$.
        \item Si $u$ est de la forme $(1,0)$, mettre $u$ de la forme $(1,1)$.
      \end{itemize}
    \item Transformer $v$ de la même façon.
    \end{itemize}
  \item Calculer le coût de la solution optimale $S$ à l'aide de l'algorithme dans le cas linéaire.
  \item Si $\Upsilon_{G(e_0,u,v)} + c(\tau_{u,v}) < \mbox{Coût}(S_{min})$, remplacer les caractéristiques de $S_{min}$ par celles de $S$ (c'est-à-dire son coût, l'arête $e_0$ et les sommets de départ $u$ et d'arrivée $v$).
  \end{enumerate}
\item Pour $e_0$, $u$ et $v$ caractérisant la solution optimale,
  \begin{enumerate}
  \item Calculer l'ensemble de la solution (donc des mouvements) à l'aide de l'algorithme dans le cas linéaire dans le graphe $\bs{G(e_0,u,v)}$.
  \item Ajouter le trajet $\tau_{u,v}$ de $u$ à $v$ en fin de solution.
  \item \begin{enumerate}
    \item Si $p$ était initialement en excès, choisir un instant du cycle de la solution où le camion arrive vide en $p$ et repart plein et dérouler les mouvements de la solution.
    \item Si $p$ était initialement en défaut, choisir un instant du cycle de la solution où le camion arrive plein en $p$ et repart vide et dérouler les mouvements de la solution.
    \end{enumerate}
  \end{enumerate}
\end{enumerate}


\section{Preuve de l'optimalité de l'algorithme}
\label{Capicité unitaire preuve}

La recherche de la solution optimale dans le graphe circulaire $G=(V,E)$ revient à chercher le plus court circuit hamiltonien dans le graphe biparti complet $G'=(V,E')$ défini et pondéré comme suit. Pour $u,v \in V$ avec $u \ne v$,
\begin{itemize}
\item si $u$ et $v$ sont toutes deux en excès ou toutes deux en défaut, on n'ajoute pas l'arête $(u,v)$.
\item sinon, on ajoute l'arête $(u,v)$ et on définit le coût $c_{(u,v)}$ de l'arête par  $c_{(u,v)}~=~c(\tau_{u,v})$.
\end{itemize}
Par conséquent, une solution optimale au SSBP dans $G$ ne dépend pas du point de départ du camion.

On se donne une solution optimale $S$ au SSBP dans le graphe $G$ et on note $S'$ la solution associée dans $G'$.

Lorsque l'arête $e_0$ est traversée, on regarde dans la solution $S'$ entre quel sommet était fait le mouvement. il suffit donc de prendre le sommet d'arrivée $u$ comme sommet de départ et le sommet de départ $v$ comme sommet d'arrivé et de résoudre dans le graphe linéaire $\bs{G(e_0,u,v)}$


\section{Cas où la solution optimale est bornée par $2\sum_{e \in E}c_e$}

\begin{prop}
Sous les hypothèses de la section~\ref{Capacité unitaire - hypothèse}, si le coût $\Upsilon_G$ de la solution optimale est strictement majoré par $2\sum_{e \in E}c_e$, alors l'algorithme de la section~\ref{capacité unitaire - conjecture et algorithme} donne la solution optimale et son coût en temps polynomial.
\end{prop}

\begin{proof}
Soit $S$ une solution optimale. Par l'absurde, si $S$ passe au moins deux fois par chaque arête, alors $\Upsilon_G~\ge~2\sum_{e \in E}c_e$, ce qui est contradictoire. Donc il existe une arête $e_0$ telle que le camion passe au plus une fois par $e_0$ au cours de ce trajet. Selon le théorème~\ref{thm: capacité unitaire - optimalité}, l'algorithme de la section~\ref{capacité unitaire - conjecture et algorithme} donne la solution optimale et son coût en temps polynomial.
\end{proof}

\begin{rmq}
La condition $\Upsilon_G~<~2\sum_{e \in E}c_e$ est une condition très forte. En effet, dans le pire des cas où les $n$ sommets $(1,0)$ suivent les $n$ sommets $(0,1)$, le coût $\Upsilon_G$ de la solution optimale est de l'ordre de $n^2\min_{e \in E}c_e$ alors que la condition précédente est linéaire en $n$.
\end{rmq}