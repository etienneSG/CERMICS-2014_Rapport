\chapter*{Résumé - Abstract}
\addcontentsline{toc}{chapter}{Résumé - Abstract}

\section*{Résumé}

\textbf{Mots clés :} mathématiques théoriques, complexité, optimisation, graphe, équilibrage
\\

Ce travail s'intéresse au problème du voyageur de commerce avec capacité. Il fait suite à un premier article de P. Benchimol et al. s'intéressant à l'équilibrage d'un graphe et à sa complexité. Le cas du graphe quelconque étant NP-difficile et le cas d'un arbre étant polynomial, ce travail porte sur le cas du graphe circulaire. Après avoir démontré quelques résultats dans le cas général des graphes circulaires, nous donnerons une solution dans le cas triangulaire et une solution polynomiale dans le cas d'une capacité de transport infinie. Enfin, nous émettrons une conjecture qui permettra de résoudre en temps polynomial un cas particulier de la capacité de transport unitaire.
\\[1cm]

\section*{Abstract}

\textbf{Keywords:} theoretical mathematics, complexity, optimization, graph, balancing.
\\

This work studies a version of the C-delivery Transportation Service Provider. It follows a first article by P. Benchimol and al. dealing with the balancing of a graph and with its complexity. For any graph, the problem is NP-hard and in the case of a tree, it is polynomial. This works studies the case of circular graph. After having proved some general results about circular graphs, we will give a solution in the case of a triangular graph and a polynomial solution in the case of an infinite capacity. Last, we will formulate a conjecture which enables to solve in polynomial time a specific case of unit capacity.
