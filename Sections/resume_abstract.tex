\chapter*{Résumé - Abstract}
\addcontentsline{toc}{chapter}{Résumé - Abstract}

\section*{Résumé}

\textbf{Mots clés :} mathématiques théoriques, complexité, optimisation, graphe, équilibrage
\\

Ce travail s'intéresse au problème du voyageur de commerce avec capacité. Il fait suite à un premier article de P. Benchimol et al. s'intéressant au rééquilibrage d'un graphe et à sa complexité. Le cas du graphe quelconque étant NP-difficile et le cas d'un arbre étant polynomial, nous travaillerons sur le cas du graphe circulaire. Quelques résultats généraux dans le cas général des des graphes circulaires seront démontrés ainsi qu'une solution dans le cas triangulaire et qu'une solution polynomiale dans le cas d'une capacité de transport infinie. Enfin, un cas particulier de la capacité unitaire sera polynomial si une conjecture nécessaire à la démonstration est démontrée dans des travaux ultérieurs.
\\[1cm]

\section*{Abstract}

\textbf{Keywords :}
