\chapter*{Résumé - Abstract}
\addcontentsline{toc}{chapter}{Résumé - Abstract}

\section*{Résumé}

\textbf{Mots clés :} mathématiques théoriques, complexité, optimisation, graphe, équilibrage
\\

Ce travail s'intéresse au problème du voyageur de commerce avec capacité. Il fait suite à un premier article de P. Benchimol et al. s'intéressant à l'équilibrage d'un graphe et à sa complexité. Le cas du graphe quelconque étant NP-difficile et le cas d'un arbre étant polynomial, ce travail porte sur le cas du graphe circulaire. Après avoir démontré quelques résultats dans le cas général des graphes circulaires, nous donnerons une solution dans le cas triangulaire et qu'une solution polynomiale dans le cas d'une capacité de transport infinie. Enfin, nous émettrons une conjecture qui permettra de résoudre en temps polynomial un cas particulier de la capacité de transport unitaire.
\\[1cm]

\section*{Abstract}

\textbf{Keywords :}
