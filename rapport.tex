% Template de rapport de stage scientifique par AT016
% Lire le fichier au fur et à mesure et remplacer par vos valeurs à vous :)
% Au fait ceci est un commentaire

\documentclass[twoside,11pt,openany,a4paper]{rapport}

\graphicspath{{images/}}

\begin{document}

% Non affiché mais sera inséré dans les propriétés du fichier
\title{Complexité du problème de régulation des systèmes de vélo en libre-service sur un ring}
\author{Etienne de \textsc{Saint Germain}}
\date{\today}

\mainmatter

% Page de garde
\begin{titlepage}
  \begin{center}
    \includegraphics[scale=0.5]{logo_enpc.jpg}
    
    \vspace{0.3cm}
    \institute{École des Ponts ParisTech}\\
    \institute{CERMICS}
    
    \vspace{0.7cm}
    2013\\
    Rapport de stage scientifique
    
    \vspace{0.3cm}
    Etienne de \textsc{Saint Germain}\\
    Élève ingénieur\\
    Département IMI
    
    \vspace{2cm}
    {\Huge{\textbf{Complexité du problème de régulation des systèmes de vélo en libre-service sur un ring}}}
    
    \vfill
    {\huge{Stage réalisé au sein du CERMICS}}\\
    École des Ponts ParisTech\\
    6 et 8 avenue Blaise Pascal\\
    Cité Descartes - Champs sur Marne\\
    77455 Marne la Vallée Cedex 2\\

    \vspace{0.7cm}
    Stage effectué du 2 juin au 22 août 2014
    
    \vspace{0.3cm}
    Maître de stage: M. Frédéric Meunier

  \end{center}
\end{titlepage}

\cleardoublepage

\chapter*{Fiche de Synthèse}
\addcontentsline{toc}{chapter}{Fiche de Synthèse}

\begin{itemize}
\item Type de stage : stage scientifique
  \\
\item Année académique : 2013/2014
  \\
\item Auteur : Etienne de \textsc{Saint Germain}
  \\
\item Formation : 2ème année
  \\
\item Titre du rapport : Complexité du problème de régulation des systèmes de vélo en libre-service sur un ring
  \\
\item Organisme d’accueil : CERMICS
  \\
\item Pays d’accueil : France
  \\
\item Maître de stage : M. Frédéric Meunier
  \\
\item Mots-clés caractérisant votre rapport (4 à 5 mots maximum) : compléxité, graphe, régulation
  \\
\item Thème École : Mathématiques
\end{itemize}


\chapter*{Remerciements}
\addcontentsline{toc}{chapter}{Remerciements}

INSERT BULLSHIT HERE

\chapter*{Résumé}
\addcontentsline{toc}{chapter}{Résumé}
Résumer son rapport ici\\
\\
\textbf{Mots clés :} comment, ecrire, un, beau, rapport
\chapter*{Abstract}
\addcontentsline{toc}{chapter}{Abstract}
La même chose mais en anglais\\
\\
\textbf{Keywords :} how, to, write, bullshit

% Ces noms ne sont pas automatiques, les changer manuellement si vous passez en anglais.
\tableofcontents
\addcontentsline{toc}{chapter}{Table des matières}
\listoffigures
\addcontentsline{toc}{chapter}{Liste des figures}

\chapter*{Organisme d'accueil}
\addcontentsline{toc}{chapter}{Organisme d'accueil}

Vous devez présenter l’organisme qui vous accueille : le lieu et le contexte du stage, le type d’organisme (laboratoire public, entreprise en France, à l’étranger), sa structure, ses activités et ses ressources. Vous devez également présenter votre maître de stage dans le cadre de ses fonctions et des travaux de recherche qu’il mène.

\part{Rapport}

\chapter{Introduction}

\section{Motivations et modèle}

\subsection{Motivations et résultats précédents}
Ce stage s'inscrit dans l'étude des moyens de transport en libre service tel \emph{Vélib'} ou \emph{Autolib'} à Paris. Le principe est le suivant : un utilisateur prend un véhicule dans n'importe quelle station et se déplace jusqu'à une n'importe quelle station où il dépose son véhicule. Un des problèmes auquel est confronté ce service est le taux de foisonnement, c'est-à-dire le fait que chaque station doit garder un bon équilibre entre le nombre de places et le nombre de véhicule. L'étude de ce rééquilibrage sous certaines hypothèse a fait l'objet de recherche et certains résultat ont déjà été publié.
\\

Dans l'article \cite{Benchimol2011}, les auteurs ont introduit une version du \emph{C-delivery TSP}\footnote{littéralement, problème du voyageur de commerce avec livraison et une capacité de transport égale à $C$} défini par Chalasani et Motwani. Le modèle utilisé est nommé \emph{Static Stations Balancing Problem} abrégé SSBP. Ce modèle s'appuie sur un graphe connexe dans lequel un véhicule (que nous appelerons le camion dans la suite) déplace les véhicules en libre service (que nous nommerons vélos) d'un sommet à un autre. Le camion peut transporter un nombre fini et maximal $C$ de vélos, en déposer tout ou une partie à chaque sommet et en prendre dans la limite de sa capacité. L'objectif est de trouver le plus court trajet du camion rééquilibrant le graphe.

Les auteurs ont montré que dans le cas général (capacité $C$ et graphe quelconque), le problème était NP-difficile. Cependant, dans le cas du graphe complet avec des coûts unitaires, ils ont pu obtenir un algorithme donnant une solution au plus deux fois plus coûteuse que la solution optimale.Ils ont également décrit une borne inférieure du coût de la solution optimale\footnote{Cette description et les notations associées seront reprise dans la section \ref{Borne inf générale} afin d'en réutiliser le résultat et le notations.}. Enfin, si le graphe est un arbre, ils ont exhibé un algorithme linéaire donnant le coût d'une solution optimale et le premier mouvement d'une telle solution.
\\

Les auteurs ont également laissé plusieurs questions ouvertes. En particulier, la complexité du rééquilibrage dans le cas où le graphe est un cercle (par exemple des stations le long d'une route entourant un parc) est à ce jour non résolue. Cette question particulière est l'objet de ce stage.

\subsection{Notations et outils}

Soient $F$ un ensemble fini et $H$ un demi-groupe additif. Soit $w \in H^F$. Alors, pour tout sous-ensemble $F'$ de $F$, on note $$w(F') = \sum_{f \in F'} w(f)$$

Soit $G=(V,E)$ un graphe. Pour tout sous-ensemble de sommet $U$ de $V$, on note $\delta(U)$ l'ensemble des arrêtes ayant exactement un sommet dans $U$. Par abus de notation, si $v$ est un sommet, on notera $\delta(v)$ au lieu de $\delta(\{v\})$.

\subsection{Modèle}

On se donne un graphe $G=(V,E)$ et une capacité $C \ge 0$. Un état \emph{s} est un couple $(\bs{x},p)$ où $\bs{x} \in \RR^V_+$ et $p$ est une arrête dans $V$. Deux états $s=(\bs{x},p)$ et $s'=(\bs{y},q)$ sont dits \emph{adjacents} s'ils ont simultanément :
\begin{itemize}
\item $x_v=y_v$ pour tout $v \notin \{p,q\}$ ;
\item $pq \in E$ ;
\item $x_p-y_p = y_q-x_q$ ;
\item $\left| x_p-y_p \right| \le C$.
\end{itemize}

Un mouvement consiste à aller d'un état $s$ à un état adjacent $s'$. Si le graphe est doté par un coût $\bs{c} \in \RR^E_+$, le coût d'un mouvement entre deux états adjacents $s=(\bs{x},p)$ et $s'=(\bs{y},q)$ est simplement $c(pq)$. Le coût d'une séquence de mouvement est la somme des coût des mouvement de la séquence.
\\

Il est facile de voir que l'état $s=(\bs{x},p)$ code la position $p$ du camion et le nombre $x_v$ de vélos sur chaque stations $v$. Un mouvement correspond bien à un mouvement réel du camion pendant lequel il transorte $x_p-y_p$ vélos de la stations $p$ à la sation $q$.

Dans notre étude, le problème consiste à aller d'un état initial $i=(\bs{x},p)$ à un état final $t=(\bs{y},q)$ en utilisant que des mouvements entre des états adjacents et avec une séquence de coût minimal.

Nous dirons qu'un sommet $v$ (ou une station) est \emph{en excès} (resp. \emph{en défault}) si $x_v > y_v$ (resp. $x_v < y_v$). Une station sera dite \emph{équilibrée} si $x_v=y_v$. L'ensemble des stations équlibrées sera noté $B(\bs{x},\bs{y})$.
\\
La même notion peut être étendue à un sous-ensemble de sommets : $U \subseteq V$ est dit en excès (resp. en défault, équilibré) si $x(U) > y(U)$ (resp. $x(U) < y(U)$, $x(U) = y(U)$).
\\

Dans l'ensemble de cette étude, nous supposerons également que $x(V) = y(V)$ sinon le problème n'a pas de solutions.
\\

Le problème introduit par les auteurs de l'article \cite{Benchimol2011} s'exprime de la manière qui suit.
\\
\textbf{Données :} Un graphe $G=(V,E)$, une fonction de coût $\bs{c} \in \RR^E_+$, une capacité $C$ et deux états $i=(\bs{x},p)$ (état initial) et $t=(\bs{y},q)$ (état cible).
\\
\textbf{Tâche :} Trouver le coût minimal de la séquence de mouvements permettant d'aller de l'état $i$ à l'état $t$ et le premier mouvement d'une telle séquence.

\section{Résultats}

La section \ref{Cas du Triangle} présente un algorithme permettant d'obtenir une solution optimale en distinguant au plus sept cas dans le cas du triangle ainsi qu'une preuve de l'optimalité de la solution retournée par l'agorithme.

\chapter{Résultats préliminaires dans un graphe circulaire à \emph{n} sommets}

\section{Borne inférieure de la solution optimale (valable pour un graphe quelconque)}
\label{Borne inf générale}

Nous nous contentons ici de récrire la borne inférieure trouvées par les auteurs de l'article \cite{Benchimol2011}.
\\

Une borne inféieure du SSBP est la solution optimale du programme linéaire en nombre entier dont les variables ``comptent'' le nombre de fois où le camion passe par chaque arrête. Les contraintes sont les suivantes :
\begin{enumerate}[label=(\roman*)]
\item les variables sont des entiers naturels.
\item la condition d'Euleur : sauf peut-être en $p$ et en $q$, le camion entre et sort un nombre paire de fois.
\item ``subtour elimination'' : si le camion est en $p \in U \subseteq V $ et qu'il existe des stations non-équilibrée dans $\bar{U}$, alors le camion doit nécessairement traverser $\delta(U)$ au moins une fois (voire plus suivant la position de $q$). Nous utiliserons la notation suivante pour écrire cette contrainte :
\[
\mu(p,q,U,\bs{x},\bs{y}) = \left\{
\begin{array}{ll}
  0 &\text{ si } p,q \in U \text{ et } \bar{U} \subseteq B(\bs{x},\bs{y})\\
  0 &\text{ si } p,q \in \bar{U} \text{ et } U \subseteq B(\bs{x},\bs{y})\\
  1 &\text{ si } p \in U \text{ et } q \in \bar{U}\\
  1 &\text{ si } p \in \bar{U} \text{ et } q \in U\\
  2 &\text{ si } p,q \in U \text{ et } \bar{U} \backslash B(\bs{x},\bs{y}) \ne \emptyset\\
  2 &\text{ si } p,q \in \bar{U} \text{ et } U \backslash B(\bs{x},\bs{y}) \ne \emptyset
\end{array}
\right.
\]
\item Contrainte de capacité : si $U \subseteq V$ a trop de vélos, le camion doit quitter suffisament de fois $U$ afin de sortir tous ces vélo. Une notation utile est la suivante :
\[
\nu(p,q,U) = \left\{
\begin{array}{ll}
  -1 &\text{ si } p \in U \text{ et } q \in \bar{U}\\
  0  &\text{ si } p \in U \text{ et } q \in U\\
  0  &\text{ si } p \in \bar{U} \text{ et } q \in \bar{U}\\
  +1 &\text{ si } p \in \bar{U} \text{ et } q \in U\\
\end{array}
\right.
\]
\end{enumerate}

Cette borne inférieure est donc solution du programme linéaire en nombre entier
\[
\begin{array}{llll}
  \text{Min}_{\bs{z}} &\sum_{e \in E} c_ez_e & & \\
  \text{s.c.}       &z_e \in \ZZ_+ &\text{pour tout } e \in E &\text{(i)} \\
                    &z(\delta(v)) \text{ est paire} &\text{pour tout } v \in V &\text{(ii)} \\
                    &z(\delta(U)) \ge \mu(p,q,U,\bs{x},\bs{y}) &\text{pour tout } U \subseteq V, U \ne \emptyset &\text{(iii)} \\
                    &z(\delta(U)) \ge \mu(p,q,U,\bs{x},\bs{y}) &\text{pour tout } U \subseteq V, U \ne \emptyset &\text{(iii)}
\end{array}
\]

\section{Un exemple où la borne inférieure n'est pas atteinte}

\chapter{Cas du triangle}
\label{Cas du Triangle}

\section{Rappel des notations et des hypothèses}

\textbf{*Insérer une figure du graphe triangulaire*}


\section{Algorithme d'obtention de la solution optimale}

\begin{thm}
\emph{Optimalité de l'algorithme précédent}

L'algorithme précédent donne le premier mouvement d'une solution optimale et le coût de cette solution est donnée par :

$$\sum_{v \in V}\xi_v(p,q,\bf{x},\bf{y})$$
\end{thm}

\section{Preuve de l'algorithme précédent}

\chapter{Capacité du camion infinie}

\chapter{Capacité du camion unitaire}

\chapter{Questions ouvertes}

\chapter*{Bilan personnel}
\addcontentsline{toc}{chapter}{Bilan personnel}
Il est important de souligner vos acquis personnels et professionnels durant le stage et de  dresser un bilan de cette expérience scientifique, humaine et linguistique le cas échéant. Expliquez quelle peut être l’influence de ce stage sur la poursuite de vos études et de votre projet professionnel.

\bibliographystyle{siam}
\bibliography{rapport}
\addcontentsline{toc}{chapter}{Bibliographie}

\backmatter

\appendix
\part{Annexes}
\chapter{Annexe 1}
Ecrire ici

\chapter{Annexe 2}
Ou là


\end{document}
