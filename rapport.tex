% Template de rapport de stage scientifique par AT016
% Lire le fichier au fur et à mesure et remplacer par vos valeurs à vous :)
% Au fait ceci est un commentaire

\documentclass[twoside,10pt,openany,a4paper]{rapport}

\graphicspath{{images/}}

\begin{document}

% Non affiché mais sera inséré dans les propriétés du fichier
\title{Complexité du problème de régulation des systèmes de vélo en libre-service sur un ring}
\author{Etienne de \textsc{Saint Germain}}
\date{\today}

\mainmatter

% Page de garde
\begin{titlepage}
  \begin{center}
    \includegraphics[scale=0.5]{logo_enpc.jpg}
    
    \vspace{0.3cm}
    \institute{École des Ponts ParisTech}\\
    \institute{CERMICS}
    
    \vspace{0.7cm}
    2013\\
    Rapport de stage scientifique
    
    \vspace{0.3cm}
    Etienne de \textsc{Saint Germain}\\
    Élève ingénieur\\
    Département IMI
    
    \vspace{2cm}
    {\Huge{\textbf{Complexité du problème de régulation des systèmes de vélo en libre-service sur un ring}}}
    
    \vfill
    {\huge{Stage réalisé au sein du CERMICS}}\\
    École des Ponts ParisTech\\
    6 et 8 avenue Blaise Pascal\\
    Cité Descartes - Champs sur Marne\\
    77455 Marne la Vallée Cedex 2\\

    \vspace{0.7cm}
    Stage effectué du 2 juin au 22 août 2014
    
    \vspace{0.3cm}
    Maître de stage: M. Frédéric Meunier

  \end{center}
\end{titlepage}

\cleardoublepage

\chapter*{Fiche de Synthèse}
\addcontentsline{toc}{chapter}{Fiche de Synthèse}

\begin{itemize}
\item Type de stage : stage scientifique
  \\
\item Année académique : 2013/2014
  \\
\item Auteur : Etienne de \textsc{Saint Germain}
  \\
\item Formation : 2ème année
  \\
\item Titre du rapport : Complexité du problème de régulation des systèmes de vélo en libre-service sur un ring
  \\
\item Organisme d’accueil : CERMICS
  \\
\item Pays d’accueil : France
  \\
\item Maître de stage : M. Frédéric Meunier
  \\
\item Mots-clés caractérisant votre rapport (4 à 5 mots maximum) : compléxité, graphe, régulation
  \\
\item Thème École : Mathématiques
\end{itemize}


\chapter*{Remerciements}
\addcontentsline{toc}{chapter}{Remerciements}

INSERT BULLSHIT HERE

\chapter*{Résumé}
\addcontentsline{toc}{chapter}{Résumé}
Résumer son rapport ici\\
\\
\textbf{Mots clés :} comment, ecrire, un, beau, rapport
\chapter*{Abstract}
\addcontentsline{toc}{chapter}{Abstract}
La même chose mais en anglais\\
\\
\textbf{Keywords :} how, to, write, bullshit

% Ces noms ne sont pas automatiques, les changer manuellement si vous passez en anglais.
\tableofcontents
\addcontentsline{toc}{chapter}{Table des matières}
\listoffigures
\addcontentsline{toc}{chapter}{Liste des figures}

\chapter*{Organisme d'accueil}
\addcontentsline{toc}{chapter}{Organisme d'accueil}

Vous devez présenter l’organisme qui vous accueille : le lieu et le contexte du stage, le type d’organisme (laboratoire public, entreprise en France, à l’étranger), sa structure, ses activités et ses ressources. Vous devez également présenter votre maître de stage dans le cadre de ses fonctions et des travaux de recherche qu’il mène.

\part{Rapport}

\chapter{Introduction}

\section{Motivations et modèle}

\subsection{Motivations et résultats précédents}
Ce stage s'inscrit dans l'étude des moyens de transport en libre service tel \emph{Vélib'} ou \emph{Autolib'} à Paris. Le principe est le suivant : un utilisateur prend un véhicule dans n'importe quelle station et se déplace jusqu'à une n'importe quelle station où il dépose son véhicule. Un des problèmes auquel est confronté ce service est le taux de foisonnement, c'est-à-dire le fait que chaque station doit garder un bon équilibre entre le nombre de places et le nombre de véhicule dans chaque station. L'étude de ce rééquilibrage sous certaines hypothèse est l'objet de ce stage.
\\

Dans l'article \cite{Benchimol2011}, les auteurs ont introduit une version du \emph{C-delivery TSP}\footnote{littéralement, problème du voyageur de commerce avec livraison et une capacité de transport égale à \it{C}} défini par Chalasani et Motwani. Le \emph{Static Stations Balancing Problem} abrégé SSBP est un 

\subsection{Notations et outils}

\subsection{Modèle}

\textbf{Données :} Un graphe $G=(V,E)$, une fonction de coût $\bf{c} \in \RR^E_+$, une capacité $C$ et deux états $i=(\bf{x},p)$ (état initial) et $t=(\bf{y},q)$ (état cible).

\textbf{Tâche :} Trouver le coût minimal de la séquence de mouvements permettant d'aller de l'état $i$ à l'état $t$ et le premier mouvement d'une telle séquence.

\section{Résultats}

\chapter{Résultats préliminaires dans un graphe circulaire à \emph{n} sommets}

\chapter{Cas du triangle}

\section{Rappel des notations et des hypothèses}

\textbf{*Insérer une figure du graphe triangulaire*}


\section{Algorithme d'obtention de la solution optimale}

\begin{thm}
\emph{Optimalité de l'algorithme précédent}

L'algorithme précédent donne le premier mouvement d'une solution optimale et le coût de cette solution est donnée par :

$$\sum_{v \in V}\xi_v(p,q,\bf{x},\bf{y})$$
\end{thm}

\section{Preuve de l'algorithme précédent}

\chapter{Capacité du camion infinie}

\chapter{Capacité du camion unitaire}

\chapter{Questions ouvertes}

\chapter*{Bilan personnel}
\addcontentsline{toc}{chapter}{Bilan personnel}
Il est important de souligner vos acquis personnels et professionnels durant le stage et de  dresser un bilan de cette expérience scientifique, humaine et linguistique le cas échéant. Expliquez quelle peut être l’influence de ce stage sur la poursuite de vos études et de votre projet professionnel.

% Inclure ici ceux qui n'ont pas été cités dans le document
\nocite{Merchet2007}
\nocite{Goossens1993}
\nocite{Greenwade1993}

\bibliographystyle{siam}
\bibliography{rapport}
\addcontentsline{toc}{chapter}{Bibliographie}

\backmatter

\appendix
\part{Annexes}
\chapter{Annexe 1}
Ecrire ici

\chapter{Annexe 2}
Ou là


\end{document}
